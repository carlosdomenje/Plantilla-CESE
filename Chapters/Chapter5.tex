% Chapter Template

\chapter{Conclusiones} % Main chapter title

\label{Chapter5} % Change X to a consecutive number; for referencing this chapter elsewhere, use \ref{ChapterX}

En este capítulo se presentan los aspectos más relevantes del trabajo realizado y
se mencionan los pasos a seguir.

%----------------------------------------------------------------------------------------

%----------------------------------------------------------------------------------------
%	SECTION 1
%----------------------------------------------------------------------------------------

\section{Trabajo realizado}

Se desarrolló e implementó un sistema de gestión de dispositivos conversores de protocolo Modbus a MQTT conectados a sensores de temperatura. La plataforma web demostró ser útil para el análisis del comportamiento de los sensores frente a las variaciones diarias de uso y aplicación.  A continuación se listan los logros destacados del trabajo final:

\begin{itemize}
	\item Programación e implementación de software en el servidor de datos para la vinculación de dispositivos conversores.
	\item Implementación de certificados SSL para dotar de seguridad a todo el sistema.
	\item Desarrollo de base de datos para el almacenamiento histórico de datos enviados para su posterior análisis. 
	\item Implementación de aplicación web para visualización, análisis y control de datos enviados por diferentes dispositivos conversores.
	\item Integración en la nube del sistema de gestión. 
\end{itemize}

El grado de cumplimiento de los requerimientos fué como se tenía previsto durante la planificación ya que se pudo lograr integrar el sistema e instalarlo en un servidor remoto para realizar pruebas con clientes. Estos últimos se encuentran ensayando los dispositivos conversores de protocolo Modbus a MQTT conectados a sensores de temperatura para el monitoreo del funcionamiento de cámaras frigoríficas.

Fue necesario contratar un servicio de servidor en la nube y un servicio de hosting web para poder realizar las pruebas de forma remota y que diferentes personas puedan probar el sistema. Esto llevó a un pequeño atraso en el desarrollo ya que se debió estudiar nuevos conceptos de programación y configuración de estos servicios. 

Durante el desarrollo de este trabajo final se aplicaron conocimiento adquiridos a lo largo de todo el año de la Especialización en Internet de las Cosas. Todas las asignaturas cursadas aportaron conocimientos necesarios y experiencia para la práctica profesional en el área del desarrollo web.  Sin embargo, se resaltan a continuación aquellas materias de mayor relevancia para este trabajo:

\begin{itemize}
	\item Gestión de Proyectos: la elaboración de un plan de proyecto para organizar el trabajo final, facilitó la realización del mismo y evitó demoras innecesarias. 
	\item Protocolos de Internet: se aplicaron conceptos aprendidos para la programación del servidor con los protocolos MQTT y HTTP.
	\item Desarrollo de aplicaciones multiplataforma: se adquirieron conocimientos de programación para la plataforma web adaptable a cualquier dispositivo que pueda ejecutar un navegador web.
	\item Arquitectura de datos: se desarrolló la base de datos teniendo en cuenta las especificaciones y técnicas aprendidas. 
	\item Ciberseguridad en IoT: se utilizaron técnicas de seguridad para proteger al sistema frente a posible ataques cibernéticos. 
	\item Testing de Sistemas de Internet de las Cosas: se aplicaron los conocimientos adquiridos durante la materia, sobre todo en las áreas de testing unitarios y ensayos de integración del sistema.
\end{itemize}


%----------------------------------------------------------------------------------------
%	SECTION 2
%----------------------------------------------------------------------------------------
\section{Próximos pasos}

Resulta imprescindible identificar el trabajo futuro para dar continuidad al esfuerzo realizado hasta el momento y poder realizar un sistema comercialmente atractivo. A continuación se listan las líneas de trabajo más trascendentes:

\begin{itemize}
	\item Desarrollo de notificaciones y alertas al usuario ante posible eventos configurables.
	\item Lectura de dispositivos que no pertenezcan a la medición de temperatura.
	\item Agregar un nivel de gestión de organizaciones para permitir a un usuario ser parte de varias de ellas. 
	\item Contratación de un servicio de servidor remoto de produccion para asegurar la disponibilidad, integridad y utilización de recursos del software implementado.
\end{itemize}
